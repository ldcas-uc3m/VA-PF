\section{Desarrollo}

% \subimport{}{2.1-dataset.tex}

\subsection{Influencia de los Umbrales en la Inferencia}

Viendo la figura \ref{fig:objectness_TH_SCORE} del Objectness-RPN, a medida que se aumenta el TH\textunderscore SCORE aumenta la precisión y disminuye el recall. Esto se debe a que se está variando la sensibilidad del modelo en la detección de regiones. Mientras que en la figura \ref{fig:global_TH_SCORE}, podemos ver que se mantiene estable a lo largo de los diferentes valores de TH\textunderscore SCORE, esto se debe a que nuestro modelo clasifica bien y tiene unos valores de umbral altos.


\begin{figure}[h]
\centering
\begin{minipage}[t]{.5\textwidth}
  \centering
  \resizebox{\linewidth}{!}{\includesvg{Objectness-RPN_th_score}}
  \captionsetup{font=footnotesize}
  \caption{Objectness-RPN TH\textunderscore SCORE}
  \label{fig:objectness_TH_SCORE}
\end{minipage}%
\begin{minipage}[t]{.5\textwidth}
  \centering
  \resizebox{\linewidth}{!}{\includesvg{Global-Classification_th_score}}
  \captionsetup{font=footnotesize}
  \caption{Global Classification TH\textunderscore SCORE}
  \label{fig:global_TH_SCORE}
\end{minipage}
\end{figure}


\begin{figure}[h]
\centering
\begin{minipage}[t]{.5\textwidth}
  \centering
  \resizebox{\linewidth}{!}{\includesvg{Objectness-RPN_th_iou}}
  \captionsetup{font=footnotesize}
  \caption{Objectness-RPN TH\textunderscore IOU}
  \label{fig:objectness_TH_IOU}
\end{minipage}%
\begin{minipage}[t]{.5\textwidth}
  \centering
  \resizebox{\linewidth}{!}{\includesvg{Global-Classification_th_iou}}
  \captionsetup{font=footnotesize}
  \caption{Global Classification TH\textunderscore IOU}
  \label{fig:global_TH_IOU}
\end{minipage}
\end{figure}


