\section{Desarrollo}

% \subimport{}{2.1-dataset.tex}

\subsection{Influencia de los Umbrales en la Inferencia}

Viendo la figura \ref{fig:objectness_TH_SCORE} del Objectness-RPN, a medida que se aumenta el TH\textunderscore SCORE aumenta la precisión y disminuye el recall. Esto se debe a que se está variando la sensibilidad del modelo en la detección de regiones. Mientras que en la figura \ref{fig:global_TH_SCORE}, podemos ver que se mantiene estable a lo largo de los diferentes valores de TH\textunderscore SCORE, esto se debe a que nuestro modelo clasifica bien y tiene unos valores de umbral altos.


\begin{figure}[h]
\centering
\begin{minipage}[t]{.5\textwidth}
  \centering
  \resizebox{\linewidth}{!}{\includesvg{Objectness-RPN_th_score}}
  \captionsetup{font=footnotesize}
  \caption{Objectness-RPN TH\textunderscore SCORE}
  \label{fig:objectness_TH_SCORE}
\end{minipage}%
\begin{minipage}[t]{.5\textwidth}
  \centering
  \resizebox{\linewidth}{!}{\includesvg{Global-Classification_th_score}}
  \captionsetup{font=footnotesize}
  \caption{Global Classification TH\textunderscore SCORE}
  \label{fig:global_TH_SCORE}
\end{minipage}
\end{figure}


\begin{figure}[h]
\centering
\begin{minipage}[t]{.5\textwidth}
  \centering
  \resizebox{\linewidth}{!}{\includesvg{Objectness-RPN_th_iou}}
  \captionsetup{font=footnotesize}
  \caption{Objectness-RPN TH\textunderscore IOU}
  \label{fig:objectness_TH_IOU}
\end{minipage}%
\begin{minipage}[t]{.5\textwidth}
  \centering
  \resizebox{\linewidth}{!}{\includesvg{Global-Classification_th_iou}}
  \captionsetup{font=footnotesize}
  \caption{Global Classification TH\textunderscore IOU}
  \label{fig:global_TH_IOU}
\end{minipage}
\end{figure}


\subsection{Anchors}

    Analizando las salidas del modelo, y comparando los anchors empleados 
    (su tamaño y relación de aspecto) con las bounding boxes de las imagenes proporcionadas, 
    se puede observar que la distribución de las predicciones, con respecto a la relación de aspecto 
    y el tamaño es bastante similar a la verdad objetiva.

    Debido a los anchors empleados, los objetos que mejor se detectan suelen ser 
    los que tienen bounding boxes con una relación de aspecto más cercana a uno. 
    Por lo que las sillas, que tienen bounding boxes más ''cuadradas'', se predicen mejor con este modelo.

    En cuanto al tamaño de los anchors, se predicen mejor los objetos más grandes, 
    seguramente a que la reducción del tamaño de la imagen hace que se pierda algo de información.

    Como conclusión de esta sección, se podría mejorar el modelo añadiendo 
    anchors con una relación de aspecto algo superior (3:1, o 4:1) para captar 
    los objetos en esa sección que no se están percibiendo. De igual manera, 
    se podría probar a añadir anchors de menor tamaño, para captar las botellas o de 
    mayor tamaño, para captar mejor las mesas o los sofás.

